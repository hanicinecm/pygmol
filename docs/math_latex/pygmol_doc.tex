\documentclass[12pt,a4paper]{article}

% packages
\usepackage{graphicx}
\usepackage{setspace}
\usepackage{bibentry}
\usepackage[format=hang,font=small,labelfont=bf]{caption}  % caption text formating
\usepackage{array}
\usepackage{geometry}
\usepackage[noadjust]{cite}
\usepackage{longtable}
\usepackage{tabularx}
\usepackage{mathtools}
\usepackage{amssymb}
\usepackage{hyperref}
\usepackage{amsmath}
\usepackage{lipsum}
\usepackage{fancyhdr}
\usepackage{bm}
\usepackage{indentfirst}
\usepackage{xcolor}

% some settings
\newcolumntype{L}[1]{>{\raggedright\arraybackslash}p{#1}}  % left-aligned text in fixed olumn
\newcolumntype{R}[1]{>{\raggedleft\arraybackslash}p{#1}}  % right-aligned text in fixed column
\geometry{a4paper, top=30mm, bottom=30mm, left=30mm, right=30mm}  % total is 210 x 297 mm

% hyperlinks:
\hypersetup{
    colorlinks=true,
    linkcolor=black,
    urlcolor=blue,
    citecolor=blue
}

% bold vectors notation and bold unit vectors:
\renewcommand{\vec}[1]{\mathbf{#1}}
\newcommand{\oldhat}{\hat}
\renewcommand{\hat}[1]{\oldhat{\mathbf{#1}}}

\setcounter{secnumdepth}{0}

% footer
\pagestyle{fancy}
\renewcommand{\headrulewidth}{0.0pt}
\fancyhf{}
%\setlength{\footskip}{50pt}
\cfoot{\thepage}
\lfoot{\tiny \vspace{5\baselineskip} \color{darkgray}
    Author: \href{https://github.com/hanicinecm}{Martin Hanicinec},
    PyGMol version: \href{https://pypi.org/project/pygmol/1.1.1/}{1.1.1},
    Documentation compiled on: \today
}

\begin{document}

\section*{Python Global Model Documentation}

This document describes the PyGMol global model and explicitly the equations which are being solved for inside in the
model, with the intention, to give the most complete and transparent description.

\tableofcontents
\newpage


\section{Symbols}\label{sec:symbols}
\setstretch{1.0}
\begin{longtable}[H]{L{16mm}R{25mm}L{94mm}}
    \caption{Overview of symbols used in the description of the PyGMol global modelling framework.} \\
    \label{tab:03-symbols} \\
    \noalign{\vskip -5mm}
    \hline
    \hline
    \noalign{\vskip 1mm}
    \textbf{Symbol} & \textbf{Unit} & \textbf{Description} \\
    \hline
    \noalign{\vskip 1mm}
    $i, k$ & & Index, running over species in a kinetic scheme \\
    $i_{0}, i_{+}, i_{-}$ & & Indices $i$ running over neutral, positive and negative species respectively \\
    $j$ & & Index, running over reactions in a kinetic scheme \\
    $N_{\mathrm{S}}$ & & number of species in a kinetic scheme \\
    $n_{i}$ & [m$^{-3}$] & Number density of the $i$-th species in a kinetic scheme \\
    $n_{\mathrm{e}}$ & [m$^{-3}$] & Electron number density \\
    $\varrho_{\mathrm{e}}$ & [eV$\cdot$m$^{-3}$] & Electron energy density \\
    $P$ & [W] & Absorbed power \\
    $p_{0}$ & [Pa] & Desired pressure \\
    $p$ & [Pa] & Instantaneous pressure \\
    $Q_{i}$ & [sccm] & Feed flow for $i$-th species in a kinetic scheme \\
    $R_{\mathrm{p}}, Z_{\mathrm{p}}$ & [m] & Plasma dimensions: radius and length \\
    $V$ & [m$^{3}$] & Plasma volume \\
    $\mathrm{T}_{\mathrm{g}}$ & [K] & Neutral temperature \\
    $\mathrm{T}_{\mathrm{i}}$ & [K] & Positive ion temperature \\
    $\mathrm{T}_{\mathrm{e}}$ & [K] & Electron temperature \\
    $T_{\mathrm{g}}$ & [eV] & Neutral temperature \\
    $T_{\mathrm{i}}$ & [eV] & Positive ion temperature \\
    $T_{\mathrm{e}}$ & [eV] & Electron temperature \\
    $k_{j}$ & [m$^{-3 + 3m}$s$^{-1}$] & Reaction rate coefficient of the $j$-th reaction of an order $m$ in a kinetic
                                        scheme \\
    $R_{j}$ & [m$^{-3}$s$^{-1}$] & Reaction rate of the $j$-th reaction in a kinetic scheme \\
    $M_{\mathrm{cp}, j}$ & [kg] & Mass of the collision partner in $j$-th electron reaction \\
    $M_{i}$ & [kg] & Mass of the $i$-th species \\
    $m_{\mathrm{e}}$ & [kg] & Electron mass \\
    $q_{i}$ & [e] & Charge of the $i$-th species \\
    $\mathrm{e}$ & [C] & Electron charge \\
    $\sigma^{\mathrm{LJ}}_{i}$ & [m] & $\sigma$ parameter of the Lennard-Jones potential for $i$-th species \\
    $\Delta E_{\mathrm{e}, j}^{\mathrm{inel}}$ & [eV] & Electron energy loss due to an inelastic collision $j$ \\
    $a_{ij}^{\mathrm{L}}$ & & stoichiometric coefficient of $i$-th distinct species on left hand side in
                              $j$-th reaction \\
    $a_{ij}^{\mathrm{R}}$ & & stoichiometric coefficient of $i$-th distinct species on right hand side in
                              $j$-th reaction \\
    $a_{ij}$ & & Net stoichiometric coefficient of $i$-th distinct species in $j$-th reaction \\
    $A_{j}$ & [m$^{-3 + 3m}$s$^{-1}$] & Arrhenius parameter -- pre-exponential factor \\
    $n_{j}$ & & Arrhenius parameter -- exponent \\
    $E_{\mathrm{a}, j}$ & [eV] & Arrhenius parameter -- activation energy \\
    $\mathrm{E}_{\mathrm{a}, j}$ & [K] & Arrhenius parameter -- activation energy \\
    $k_{\mathrm{B}}$ & [JK$^{-1}$] & Boltzmann constant \\
    $s_{i}$ & & Sticking coefficient -- probability of $i$-th species sticking to a plasma boundary ($s_{i} \in
                [0, 1]$ ) \\
    $r_{ik}$ & & Return coefficient -- number of $i$-th species returned for each \emph{one} of \emph{stuck} $k$-th
                 species ($r_{ik} \in \mathbf{R}^{+}_{0}$) \\
    $D_{i}$ & [m$^{2}\mathrm{s}^{-1}$] & Diffusion coefficient of $i$-th species \\
    $D_{\mathrm{a}}$ & [m$^{2}\mathrm{s}^{-1}$] & Ambipolar diffusion coefficient \\
    $\Lambda$ & [m] & Characteristic diffusion length \\
    $\lambda_{i}$ & [m] & Mean free path of $i$-th species \\
    $\overline{v}_{i}$ & [ms$^{-1}$] & Mean speed of $i$-th species \\
    $\sigma^{\mathrm{m}}_{ik}$ & [m$^{2}$] & Momentum transfer cross section for $i$-th species scattering on
                                             $k$-th species \\
    $\overline{V_{\mathrm{s}}}$ & [V] & Mean sheath voltage \\
    $n_{\min}$ & [m$^{-3}$] & Minimal allowed particle density \\
    $\varrho_{\mathrm{e}, \min}$ & [eV$\cdot$m$^{-3}$] & Minimal allowed electron energy density \\
    \hline
    \hline
\end{longtable}
\setstretch{1.5}

\section{Model Description}
The model solves the set of following ordinary differential equations (ODE):
\begin{itemize}
    \item[--]
    \textbf{Particle density balance equation for heavy species}, including contributions from volumetric reactions,
    flow and from diffusion sinks and surface sources of $ n_{i} $.
    \item[--]
    \textbf{Electron energy density balance equation}, including contributions of power absorbed by the plasma,
    elastic and inelastic collisions between electrons and heavy species, generation and loss of electrons in
    volumetric reactions and power lost to walls by electrons and ions.
\end{itemize}

The electron density $n_{\mathrm{e}}$ is not solved for explicitly but rather implicitly by enforcing the charge
neutrality.
Also, the heavy species temperature is not resolved in the model, but rather treated as a constant input parameter.
The collisional kinetics is not described by cross sections, but rather parametrized for each reaction with the
Arrhenius formula~\eqref{eq:03-arrh-1},~\eqref{eq:03-arrh-2}.
The model was developed mainly for the purpose of plasma chemistry reduction, which justifies it's simplicity and the
degree of approximation.
For those reasons, the model should only be used with a great care to obtain any sort of quantitative results.

\subsection{Input parameters for the model:}
\begin{itemize}
    \item[--] Plasma parameters:
    $P$, $p_{0}$, $Q_{i}$, $R_{\mathrm{p}}$, $Z_{\mathrm{p}}$, $\mathrm{T}_{\mathrm{g}}$
    \item[--] Reaction set parameters:
    $M_{i}$, $q_{i}$, $\sigma^{\mathrm{LJ}}_{i}$, $k_{j}$, $\Delta E_{\mathrm{e}, j}^{\mathrm{inel}}$
\end{itemize}

\subsection{Model outputs:}
\begin{itemize}
    \item[--] Densities of heavy species and electrons: $n_{i}$, $n_{\mathrm{e}}$
    \item[--] Electron temperature $T_{\mathrm{e}}$
\end{itemize}
The rest of this chapter describes all the solved equations in detail.

\section{Particle Density Balance}\label{sec:03-particle-density-balance}
The time derivative of all heavy species densities is expressed as a sum of contributions from volumetric processes,
flow sources and sinks and surface (diffusion) processes:
\begin{equation}
    \frac{\mathrm{d} n_{i}}{\mathrm{d} t} =
    \left( \frac{\delta n_{i}}{\delta t} \right)_{\mathrm{vol}} +
    \left( \frac{\delta n_{i}}{\delta t} \right)_{\mathrm{flow}} +
    \left( \frac{\delta n_{i}}{\delta t} \right)_{\mathrm{diff}}.
    \label{eq:03-pbe}
\end{equation}

\subsection{Volumetric Reactions Contribution}\label{subsec:03-volumetric-reactions-contribution}
The contribution of volumetric reactions is
\begin{equation}
    \left( \frac{\delta n_{i}}{\delta t} \right)_{\mathrm{vol}} = \sum_{j} G_{ij} - \sum_{j} L_{ij},
\end{equation}
where $G_{ij}$ and $L_{ij}$ are contributions of generation and loss of $ n_{i} $ due to inelastic reaction $ j $.
In greater detail, it can be written as

\begin{equation}
    \left( \frac{\delta n_{i}}{\delta t} \right)_{\mathrm{vol}} =
    \sum_{j} (a_{ij}^{\mathrm{R}} - a_{ij}^{\mathrm{L}}) R_{j},
\end{equation}

\begin{equation}
    R_{j} = k_{j} \prod_{l} n_{lj}^{\mathrm{L}},
    \label{eq:03-react-rate}
\end{equation}
where $n_{lj}^{\mathrm{L}}$ is the density of $l^{\mathrm{th}}$ species on the left hand side of reaction $j$.
The reaction rate coefficient $k_{j}$ in this model takes form of the Arrhenius equation

\begin{equation}
    k_{j} =
    A_{j} \left( \frac{\mathrm{T}_{\mathrm{g}}}{300 \mathrm{K}} \right)^{n_{j}}
    \exp \left( - \frac{\mathrm{E}_{\mathrm{a}, j}}{\mathrm{T}_{\mathrm{g}}} \right)
    \label{eq:03-arrh-1}
\end{equation}
for heavy species reactions $j$ (reactions where all the reactants are heavy species) and

\begin{equation}
    k_{j} = A_{j}
    \left( \frac{T_{\mathrm{e}}}{1 \mathrm{eV}} \right)^{n_{j}}
    \exp \left( - \frac{E_{\mathrm{a}, j}}{T_{\mathrm{e}}} \right)
    \label{eq:03-arrh-2}
\end{equation}
for electron processes $j$ (reactions where at least one reactant is an electron).
The Arrhenius parameters $A_{j}$, $n_{j}$ and $E_{\mathrm{a}, j}$ (or $\mathrm{E}_{\mathrm{a}, j}$) describe the
collisional kinetics of the model.

\subsection{Flow Contribution}\label{subsec:03-flow-contribution}
The contribution of flow to the time evolution of heavy species densities will consist of inflow and outflow terms as
well as a term regulating the pressure.
\begin{equation}
    \left( \frac{\delta n_{i}}{\delta t} \right)_{\mathrm{flow}} =
    \left( \frac{\delta n_{i}}{\delta t} \right)_{\mathrm{flow}} ^{\mathrm{in}} +
    \left( \frac{\delta n_{i}}{\delta t} \right)_{\mathrm{flow}} ^{\mathrm{out}} +
    \left( \frac{\delta n_{i}}{\delta t} \right)_{\mathrm{flow}} ^{\mathrm{reg}}
\end{equation}

\subsubsection{Inflow}
\begin{equation}
    \left( \frac{\delta n_{i}}{\delta t} \right)_{\mathrm{flow}} ^{\mathrm{in}} = \frac{Q_{i}'}{V},
\end{equation}
where
\[
Q_{i}' = \frac{N_{\mathrm{A}}}{V_{\mathrm{m}} \cdot 60} Q_{i} = 4.478 \times 10^{17} \cdot Q_{i}
\]
is the inflow expressed in [particles/sec] rather than in [sccm].
$N_{\mathrm{A}} = 6.022 \times 10^{23}~\mathrm{mol}^{-1}$ is
Avogadro constant and $V_{\mathrm{m}} = 2.241 \times 10^4~\mathrm{cm}^3 \mathrm{mol}^{-1}$ is the molar volume
for ideal gas at standard temperature and pressure.

\subsubsection*{Outflow}
The outflow term is set in such a way that only neutrals are leaving the plasma region due to the flow,
the neutral species flow rate is proportional to the species density and the total flow rate out of the plasma region
is the same as total inflow rate:
\begin{equation}
    \left( \frac{\delta n_{i}}{\delta t} \right)_{\mathrm{flow}} ^{\mathrm{out}} =
    \begin{cases}
        -\dfrac{\sum  Q_{i}'}{\sum n_{i_{0}}} \cdot \dfrac{n_{i}}{V} & \mathrm{neutrals}, \\
        0 & \mathrm{ions},
    \end{cases}
\end{equation}
where the index $i_{0}$ runs only over neutral species.

\subsubsection{Pressure Regulation}
A term regulating the plasma pressure is added to the particle balance equation, accounting for changes in $p$ due
to dissociation/association processes and to diffusion losses and surface sources.
This term, similarly to the outflow term, only acts upon the neutral species and can be viewed as an addition to the
outflow term, or physically as adjusting a pressure-regulation valve between a plasma chamber and a pump,
based on the instantaneous pressure.
\begin{equation}
    \left( \frac{\delta n_{i}}{\delta t} \right)_{\mathrm{flow}} ^{\mathrm{reg}} =
    \begin{cases}
        -\dfrac{p - p_{0}}{p_{0}} \dfrac{n_{i}}{\tau_{p}} & \mathrm{neutrals}, \\
        0 & \mathrm{ions}.
    \end{cases}
\end{equation}
Here, $p$ is the instantaneous pressure from the state equation for an ideal gas
\begin{equation}
    p = k_{\mathrm{B}}T_{\mathrm{g}} \cdot \sum_{i} n_{i},
\end{equation}
and $\tau_{p}$ is a pressure recovery time scale.
A value of $\tau_p = 10^{-3}$~s was found to yield satisfactory results for a wide range of process
parameters.

\subsection{Diffusion Contribution}\label{subsec:03-diffusion-contribution}
The diffusion contribution towards the particle balance equation is ultimately controlled by vector of sticking
coefficients $s_{i}$ and matrix of return coefficients $r_{ik}$ and the diffusion model:
\begin{equation}
    \left( \frac{\delta n_{i}}{\delta t} \right)_{\mathrm{diff}} =
    \left( \frac{\delta n_{i}}{\delta t} \right)_{\mathrm{diff}}^{\mathrm{out}} +
    \left( \frac{\delta n_{i}}{\delta t} \right)_{\mathrm{diff}}^{\mathrm{in}}.
\end{equation}

\subsubsection{Diffusion losses}
The rate of species loss to the plasma boundaries due to diffusion is expressed as
\begin{equation}
    \left( \frac{\delta n_{i}}{\delta t} \right)_{\mathrm{diff}}^{\mathrm{out}} =
    \frac{A}{V} \Gamma_{\mathrm{wall}, i},
\end{equation}
where $V$ and $A$ are plasma volume and plasma boundaries area respectively, while the wall fluxes
$\Gamma_{\mathrm{wall}, i}$, as used (among others) by Kushner in GlobalKin~\cite{schroter_numerical_2018},
is expressed as
\begin{equation}
    \Gamma_{\mathrm{wall}, i} = - \frac{D_i n_i s_i}{s_i \Lambda + \frac{4 D_i}{\overline{v}_{i}}}
\end{equation}
where
\begin{equation}
    \Lambda =
    \left[
        \left( \frac{\pi}{Z_{\mathrm{p}}} \right)^{2} + \left( \frac{2.405}{R_{\mathrm{p}}} \right)^{2}
    \right]^{-1/2}.
\end{equation}

The diffusion coefficient is calculated separately for neutrals and ions.
For positive and negative ions, the diffusion coefficient is the coefficient of ambipolar diffusion in
electronegative plasma, as proposed by Stoeffels~\emph{et al.} in~\cite{stoffels_negative_1995}.
\begin{equation}
    D_{i} =
    \begin{cases}
        D_{i}^{\mathrm{free}} & \mathrm{neutrals}, \\
        D_{+}^{\mathrm{free}} \dfrac{1 + \gamma (1 + 2 \alpha)}{1 + \alpha \gamma} & \mathrm{+ions}, \\
        0 & \mathrm{-ions}.
    \end{cases}
\end{equation}
Here, $\gamma = T_{\mathrm{e}}/T_{\mathrm{i}}$ and $\alpha = \sum n_{\mathrm{i^{-}}} / n_{\mathrm{e}}$.
$D_{i} = 0$ for negative ions implies that no negative ions are reaching the plasma boundaries and therefore there
are no negative ion diffusion losses.
This approximation is justified by the positive plasma potential trapping the negative ions in the plasma bulk.
It should be noted that the stated ambipolar diffusion coefficients are only valid for the case of
$\alpha \ll \mu_{\mathrm{e}} / \mu_{\mathrm{i}}$, where $\mu$ are mobilities of electrons and ions respectively.
The free diffusion coefficient for heavy species is calculated as
\begin{equation}
    D_{i}^{\mathrm{free}} = \frac{\pi}{8} \lambda_{i} \overline{v}_{i}.
\end{equation}

The mean free path $\lambda_{i}$ for all heavy species is
\begin{equation}
    \frac{1}{\lambda_{i}} = \sum_{k}n_{k}\sigma^{\mathrm{m}}_{ik} (1-\delta_{ik})
    ,
\end{equation}
where $\sigma^{\mathrm{m}}_{ik}$ is the momentum transfer cross section, and the mean speed $\overline{v}_{i}$ is the
mean thermal speed
\begin{equation}
    \overline{v}_{i} =
    \begin{cases}
        \left( \dfrac{8 k_{\mathrm{B}} \mathrm{T}_\mathrm{g}}{\pi M_{i}} \right)^{1/2} & \mathrm{neutrals}, \\
        \left( \dfrac{8 k_{\mathrm{B}} \mathrm{T}_\mathrm{i}}{\pi M_{i}} \right)^{1/2} & \mathrm{ions}, \\
    \end{cases}
\end{equation}
where the ion temperature is approximated, as proposed by Lee and Lieberman in~\cite{lee_global_1995},
\begin{equation}
    \mathrm{T_{i}} =
    \begin{cases}
        \left( 5800 - \mathrm{T_g} \right) \dfrac{0.133}{p} + \mathrm{T_g} & p > 0.133~\mathrm{Pa}, \\
        5800 & p \le 0.133~\mathrm{Pa}. \\
    \end{cases}
\end{equation}

The momentum transfer cross section $\sigma^{\mathrm{m}}_{ik}$ is for the purpose of this model crudely approximated
with hard sphere model for neutral--neutral and ion--neutral collisions, and with momentum transfer for
Rutherford scattering (as proposed by Lieberman and Lichtenberg in~\cite{lieberman_principles_2005}) for the
case of ion--ion collisions:
\begin{equation}
    \sigma^{\mathrm{m}}_{ik} =
    \begin{dcases*}
        \left( \sigma^{\mathrm{LJ}}_{i} + \sigma^{\mathrm{LJ}}_{k} \right) ^{2} &
        $i = i_{+}, i_{-}, i_{0}$, and $ k = k_{0}$, \\
        &
        $i = i_{0}$, and $k = k_{+}, k_{-}$, \\
        \pi b_{0}^{2} \mathrm{ln}\left( \dfrac{2 \lambda_{\mathrm{De}}}{b_{0}} \right) &
        $i = i_{+}, i_{-}$, and $k = k_{+}, k_{-}$, \\
    \end{dcases*}
\end{equation}
with Debye length
\begin{equation}
    \lambda_{\mathrm{De}} = \left( \frac{\epsilon_{0} T_{\mathrm{e}}}{\mathrm{e} n_{\mathrm{e}}} \right)^{1/2},
\end{equation}
classical distance of closest approach
\begin{equation}
    b_{0} = \frac{q_{i}q_{k} \mathrm{e}^{2}}{2 \pi \epsilon_{0} m_{\mathrm{R}} v_{\mathrm{R}}^{2}},
\end{equation}
reduced mass
\begin{equation}
    m_{\mathrm{R}} = \frac{m_{i} m_{k}}{m_{i} + m_{k}},
\end{equation}
and the relative speed being approximated by the mean thermal speed
\begin{equation}
    v_{\mathrm{R}} = \overline{v}_{i}.
\end{equation}

The $\delta_{ik}$ term filters out self--collisions, as collisions between the same species do not affect the species
collective behaviour
\begin{equation}
    \delta_{ik} =
    \begin{cases}
        1 & \mathrm{for}\; i = k, \\
        0 & \mathrm{for}\; i \neq k.
    \end{cases}
\end{equation}

Finally, the free diffusion coefficient for positive ions is approximated by
\begin{equation}
    D_{+}^{\mathrm{free}} = \overline{D_{i_{+}}^{\mathrm{free}}}.
\end{equation}

\subsubsection{Boundary sources}
Each $k$-th species which is lost (or \emph{stuck}) to the plasma boundary can get returned as $i$-th species,
introducing the boundary sources
\begin{equation}
    \left( \frac{\delta n_{i}}{\delta t} \right)_{\mathrm{diff}}^{\mathrm{in}} =
    - \sum_{k} r_{ik} \left( \frac{\delta n_{k}}{\delta t} \right)_{\mathrm{diff}}^{\mathrm{out}} =
    \frac{A}{V} \sum_{k} \frac{D_k n_k s_k r_{ik}}{s_k \Lambda + \frac{4 D_k}{\overline{v}_{k}}}.
    \label{eq:03-diff-source}
\end{equation}

\subsection{Minimal Allowed Species Density}\label{subsec:03-minimal-allowed-species-density}
To prevent the ODE solver from \emph{overshooting} into unphysical negative densities, an additional \emph{artificial}
term is added to the right-hand side of~\eqref{eq:03-pbe}, ensuring a finite minimal value of particle densities
(which can be considered zero).
This correction term takes form of
\begin{equation}
    \left( \frac{\delta n_{i}}{\delta t} \right)_{n_{\mathrm{min}}} =
    \begin{cases}
        \dfrac{n_{\mathrm{min}} - n_{i}}{\tau} & n_{i} < n_{\mathrm{min}}, \\
        0 & n_{i} \ge n_{\mathrm{min}}.
    \end{cases}
\end{equation}
Here, $n_{\min}$ is set to 1~particle/m$^{3}$ and $\tau$ to $1.0 \times 10^{-10}$~s,
ensuring an adequately fast response.

\section{Electron Energy Density Balance}\label{sec:03-electron-energy-density-balance}
The balance equation for the electron energy density consists of contributions from the absorbed power,
elastic and inelastic electron collisions, electron production and consumption and contribution from diffusion losses
of electrons and ions.
\begin{equation}
    \frac{\mathrm{d} \varrho_{\mathrm{e}}}{\mathrm{d} t} =
    \frac{P}{V\mathrm{e}} -
    \left( \frac{\delta \varrho_{\mathrm{e}}}{\delta t} \right)_{\mathrm{el/inel}} -
    \left( \frac{\delta \varrho_{\mathrm{e}}}{\delta t} \right)_{\mathrm{gen/loss}} -
    \left( \frac{\delta \varrho_{\mathrm{e}}}{\delta t} \right)_{\mathrm{el \rightarrow walls}} -
    \left( \frac{\delta \varrho_{\mathrm{e}}}{\delta t} \right)_{\mathrm{ion \rightarrow walls}}
    \label{eq:03-eeb}
\end{equation}

\subsection{Contribution of Elastic and Inelastic Collisions}
\label{subsec:03-contribution-of-elastic-and-inelastic-collisions}
Electron energy density loss rate due to electron collisions is described as
\begin{equation}
    \left( \frac{\delta \varrho_{\mathrm{e}}}{\delta t} \right)_{\mathrm{el/inel}} =
    \sum_{j} R_{j} \Delta E_{\mathrm{e}, j}
    ,
\end{equation}
with the electron energy loss for $j$-th reaction $\Delta E_{\mathrm{e}, j}$ being
\begin{equation}
    \Delta E_{\mathrm{e}, j} =
    \begin{cases}
        \Delta E_{\mathrm{e}, j}^{\mathrm{inel}} & \mathrm{inelastic~collisions}, \\
        3 \dfrac{m_{\mathrm{e}}}{M_{\mathrm{cp}, j}} (T_{\mathrm{e}} - T_{\mathrm{g}}) & \mathrm{elastic~collisions}, \\
        0 & \mathrm{heavy~species~collisions~or}~a_{\mathrm{e}j}^{\mathrm{R}}=0,
    \end{cases}
\end{equation}
and
\begin{equation}
    T_{\mathrm{e}} = \frac{2}{3} \frac{\varrho_{\mathrm{e}}}{n_{\mathrm{e}}}.
\end{equation}
The electron density $n_{\mathrm{e}}$ is not resolved explicitly, but rather calculated from plasma charge neutrality
\begin{equation}
    n_{\mathrm{e}} = \sum_{i} n_{i}q_{i}.
\end{equation}
For that reason, $T_{\mathrm{e}}$ might reach non-physically low values, when $\varrho_{\mathrm{e}}$ governed directly
by~\eqref{eq:03-eeb} is much greater than $\sum_{i} n_{i}q_{i}$.
A correction is introduced to $T_{\mathrm{e}}$ in the form of
\begin{equation}
    T_{\mathrm{e}} = \max \left( T_{\mathrm{g}}, \frac{2}{3} \frac{\varrho_{\mathrm{e}}}{n_{\mathrm{e}}} \right),
\end{equation}
to help with the solution stability, as the electron temperature will converge to the neutral gas temperature at low
electron energy, due to dominance of elastic processes.

\subsection{Electron Generation and Loss Contribution}\label{subsec:03-electron-generation-and-loss-contribution}
Rate of change of electron energy density due to generation and loss of electrons is described by
\begin{equation}
    \left( \frac{\delta \varrho_{\mathrm{e}}}{\delta t} \right)_{\mathrm{gen/loss}} =
    \frac{3}{2} T_{\mathrm{e}} \sum_{j} (a_{\mathrm{e}j}^{\mathrm{R}} - a_{\mathrm{e}j}^{\mathrm{L}}) R_{j}.
\end{equation}

\subsection{Energy Loss by Electron Transport}\label{subsec:03-energy-loss-by-electron-transport}
Under a Maxwellian energy distribution assumption, each electron lost through the plasma boundary sheath takes away
$2k_{\mathrm{B}}\mathrm{T_e}$ of energy with it~\cite{lieberman_principles_2005}, which gives
\begin{equation}
    \left( \frac{\delta \varrho_{\mathrm{e}}}{\delta t} \right)_{\mathrm{el \rightarrow walls}} =
    - 2 T_{\mathrm{e}} \left( \frac{\delta n_{\mathrm{e}}}{\delta t} \right)_{\mathrm{walls}}
    ,
\end{equation}
while the total charge flux needs to be zero, yielding
\begin{equation}
    \left( \frac{\delta n_{\mathrm{e}}}{\delta t} \right)_{\mathrm{walls}} =
    \sum_{i} q_{i} \left( \frac{\delta n_{i}}{\delta t} \right)_{\mathrm{diff}}.
\end{equation}

\subsection{Energy Loss by Ion Transport}\label{subsec:03-energy-loss-by-ion-transport}
If it is assumed, that ions leave the plasma boundary sheath with the Bohm velocity, each positive ion removed from
the plasma takes away $\frac{1}{2} k_{\mathrm{B}} \mathrm{T_e}$ of kinetic energy, as well as sheath voltage
acceleration energy~\cite{lieberman_principles_2005}
\begin{equation}
    \left( \frac{\delta \varrho_{\mathrm{e}}}{\delta t} \right)_{\mathrm{ion \rightarrow walls}} =
    - \frac{1}{2} T_{\mathrm{e}} \sum_{i_{+}} \left( \frac{\delta n_{i_{+}}}{\delta t} \right)_{\mathrm{diff}}
    - \overline{V_{\mathrm{s}}} \sum_{i_{+}} q_{i_{+}} \left( \frac{\delta n_{i_{+}}}{\delta t} \right)_{\mathrm{diff}}.
\end{equation}
The last open parameter in the system is the mean sheath voltage $\overline{V_{\mathrm{s}}}$, which,
according to Lieberman and Lichtenberg~\cite{lieberman_principles_2005}, can be approximated by
\begin{equation}
    \overline{V_{\mathrm{s}}} =
    T_{\mathrm{e}} \cdot \mathrm{ln} \left( \frac{\overline{M_{i_{+}}}}{2 \pi m_{\mathrm{e}}} \right)^{1/2}.
\end{equation}
This value of $\overline{V_{\mathrm{s}}}$ is only consistent with ICP plasma sources.

\subsection{Minimal Allowed Electron Energy Density}\label{subsec:03-minimal-allowed-electron-energy-density}
To prevent the ODE solver from \emph{overshooting} into unphysical negative $\varrho_{\mathrm{e}}$, an additional
\emph{artificial} term is added to the right-hand side of~\eqref{eq:03-eeb}, ensuring a finite minimal value of electron
energy density (which can be considered zero).
This correction term takes form of
\begin{equation}
    \left( \frac{\delta \varrho_{\mathrm{e}}}{\delta t} \right)_{\varrho_{\mathrm{min}}} =
    \begin{cases}
        \dfrac{\varrho_{\mathrm{e, min}} - \varrho_{\mathrm{e}}}
              {\tau} &
        \varrho_{\mathrm{e}} < \varrho_{\mathrm{e, min}}, \\
        0 &
        \varrho_{\mathrm{e}} \ge \varrho_{\mathrm{e, min}}.
    \end{cases}
\end{equation}
Here, $\varrho_{\mathrm{e}, \min}$ was set to 1~eV/m$^{3}$ and $\tau$ to $1.0 \times 10^{-10}$~s,
ensuring an adequately fast response.



\bibliographystyle{ieeetr}
\bibliography{pygmol_doc}
\addcontentsline{toc}{section}{References}


\end{document}
